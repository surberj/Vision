\chapter*{Abstract}
\addcontentsline{toc}{chapter}{Abstract}
%\chapter*{Zusammenfassung}
%\addcontentsline{toc}{chapter}{Zusammenfassung}

Frames defined as in powerpoint.
Define:
Ground truth Trajectory: 
\begin{equation}
       \vec{q_{gt}} \vec{r_{gt}}
\end{equation}
VI-Odometry Trajectory before alignment:
\begin{equation}
       \tilde{\vec{q_{vi}}} \tilde{\vec{r_{vi}}}
\end{equation}
VI-Odometry Trajectory after alignment:
\begin{equation}
       \vec{q_{vi}} \vec{r_{vi}}
\end{equation}

Position Error

\begin{equation}
       E_{trans} \left( n \right) = \left\Vert\left(\ r_{vi} \left( n \right) - r_{gt} \left( n \right) \right) \right\Vert_2
\end{equation}

\begin{equation}
       E_{trans, avg} \left( n \right) = \frac{1}{N} \sum_{n=1}^N E_{trans} \left( n \right)
\end{equation}

\begin{equation}
       D_{vi} \left( n \right) = \sum_{n=1}^{N-1} \left\Vert\left(\ r_{vi} \left( n+1 \right) - r_{vi} \left( n \right) \right) \right\Vert_2
\end{equation}

\begin{equation}
       D_{gt} \left( n \right) = \sum_{n=1}^{N-1} \left\Vert\left(\ r_{gt} \left( n+1 \right) - r_{gt} \left( n \right) \right) \right\Vert_2
\end{equation}

\begin{equation}
       E_{trans, perc} \left( n \right) = \frac{E_{trans} \left( n \right)}{D_{gt} \left( n \right)}
\end{equation}

Rotation Error

\begin{equation}
       E_{rot} \left( n \right) = \arccos \left(\left( R_q\left(\vec{q}_{vi}\right) \vec{e_{x}} \right) \cdot \left( R_q\left(\vec{q}_{gt}\right) \vec{e_{x}} \right)\right)
\end{equation}

\begin{equation}
       E_{rot, avg} \left( n \right) = \frac{1}{N} \sum_{n=1}^N E_{rot} \left( n \right)
\end{equation}

\begin{equation}
       \vec{e_{x}} = \left[ \begin{array}{c}
	1 \\
	0 \\
	0 \\ \end{array} \right]
\end{equation}

As most VI-Odometry algorithms, ROVIO and OKVIS are tracking attitude by unit quaternions. A quaternion $ \vec{q} \in \mathbb{H} $ may be represented by a vector

\begin{equation}
       \vec{q} = \left[ \begin{array}{cccc} q_0 & q_1 & q_2 & q_3 \end{array} \right]^T 
		   = \left[ \begin{array}{c} q_0 \\ \vec{q_{1 : 3}} \end{array} \right]
\end{equation}

Rotation Matrix:

\begin{equation}
       R_q \left( \vec{q}\right) = \left[ \begin{array}{ccc}
		q_0^2+q_1^2-q_2^2-q_3^2 	& 2 q_1 q_2+2 q_0 q_3			& 2 q_1 q_3-2 q_0 q_2 \\
		2 q_1 q_2-2 q_0 q_3		& q_0^2-q_1^2+q_2^2-q_3^2 	& 2 q_2 q_3+2 q_0 q_1 \\
		2 q_1 q_3+2 q_0 q_2		& 2 q_2 q_3-2 q_0 q_1			& q_0^2-q_1^2-q_2^2+q_3^2 \\
		\end{array} \right]
\end{equation}

\begin{equation}
       \vec{q_{vi}} \vec{r_{vi}} \vec{q_{gt}} \vec{r_{gt}}
\end{equation}



\begin{tikzpicture}
% coordinate frame i
  \draw[->] (xyz cs:x=0) -- (xyz cs:x=1) node[above] {$x$};
  \draw[->] (xyz cs:y=0) -- (xyz cs:y=1) node[right] {$z$};
  \draw[->] (xyz cs:z=0) -- (xyz cs:z=1) node[above] {$y$};


%  \node[fill,circle,inner sep=1.5pt,label={left:$J$}] at (cs:x=0) {};
\end{tikzpicture}




